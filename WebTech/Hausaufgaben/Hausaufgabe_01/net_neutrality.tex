Netzneutralitaet
================

    Vorteile:
    ---------

        - Gleichberechtigung unter den Anbietern von Onlinediensten
        - Keine Kontrolle und kein Eingreifen in den Traffic durch die Provider
        - Keine Monopolisierung durch Grossunternehmen mit genug Kapital, um sich schnellere Uebertragung zu sichern (siehe Comcast und Netflix)
        - stagnation von Innovation, da neue Konzepte nicht konkurrenzfaehig sind
        - Wahrung der Meinungsfreiheit (kein Boykott von Onlinediensten)


    Nachteile:
    ----------

        - schnellere Verbindung fuer Dienste mit hohen Geschwindigkeitsanforderungen (z.B. Streaming) ist nicht gewaehrleistet
        - Netzauslastung ist hoch

    Vorschlaege zum Entgegenwirken von Verletzungen:
    ------------------------------------------------

        - politische/juristische Einflussnahme
        - Ausbau der Netzinfrastruktur mit Unterstuetzung des Staates
        - Grundsaetzliches Verbot fuer die Netzbetreiber, Pakete zu anderen Zwecken als zum Forwarden zu analysieren
        - Mechanismen zum Umgehen 

    Zero Rating:
    ------------

        - An der technischen Handhabung von Paketen aendert sich nichts
        - Dennoch werden Pakete ueber die zur Uebermittlung noetigen Informationen hinaus analysiert
        - Zur Netzneutralitaet kann auch Kostenneutralitaet zaehlen. Es treten aenliche Probleme wie bei der Geschwindigkeitsbasierten Disskusion auf:
            * Monopolisierung und Wettkampfvorteile
            * Eingriff in den Datenverkehr
            * Moeglichkeit der Boykottierung
        - Hinzu kommt, dass lediglich finanzielle Interessen im Vordergrund stehen und z.B. nicht einmal die Auslastung des Netzes gesenkt wird
